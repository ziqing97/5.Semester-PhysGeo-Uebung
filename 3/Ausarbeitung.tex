\documentclass[12pt]{article}
\usepackage{setspace,graphicx,amsmath,geometry,fontspec,titlesec,soul,bm,subfigure}
\titleformat{\section}[block]{\LARGE\bfseries}{\arabic{section}}{1em}{}[]
\titleformat{\subsection}[block]{\Large\bfseries\mdseries}{\arabic{section}.\arabic{subsection}}{1em}{}[]
\titleformat{\subsubsection}[block]{\normalsize\bfseries}{\arabic{subsection}-\alph{subsubsection}}{1em}{}[]
\titleformat{\paragraph}[block]{\small\bfseries}{[\arabic{paragraph}]}{1em}{}[]
\setmainfont{Times New Roman}
\renewcommand{\baselinestretch}{1.15}
\renewcommand\contentsname{Inhaltverzeichnis}
\geometry{a4paper,left=2.5cm,right=2.5cm,top=2.5cm,bottom=2.5cm}
\begin{document}
	\newpagestyle{main}{            
		\sethead{Ziqing Yu}{Physikalische Geodäsie Übung 3}{3218051}     
		\setfoot{}{\thepage}{}     
		\headrule                                     
		\footrule                                       
	}
	\pagestyle{main}
\tableofcontents
\newpage
\section{Superposition of Earth's and Moon's gravitational fields}
k ist in dieser Aufgabe als 1 gewählt. Die Potential und Anziehung werden nach folgende Formeln gerechnet

\[ V = \begin{cases}
\frac{4}{3} \pi G \rho R^3 \frac{1}{r} \quad (r > R)\\
2 \pi G \rho (R^2 - \frac{1}{3} r^2) \quad  (r < R)
\end{cases} \]

\[ a = \begin{cases}
- \frac{4}{3} \pi G \rho R^3 \frac{1}{r^3} \quad (r > R)\\
- \frac{4}{3} \ pi G \rho r \quad (r < R)
\end{cases}
\]

\begin{equation*}
V_{gesamt} = V_{Earth} + V_{Moon}
\end{equation*}
\begin{equation*}
a_{gesamt} = a_{Earth} + a_{Moon}
\end{equation*}
Die Darstellung von Potential und Anziehung 
\newpage
\section{Gravitational potential and attraction of spherical shells}
Die Potential und Anziehung werden in 2 Teile bzw. Kern und Mantel berechnet. 
\[ V_{c} = \begin{cases}
\frac{4}{3} \pi G \rho R_c^3 \frac{1}{r} \quad &(r > R_c)\\
2 \pi G \rho (R_c^2 - \frac{1}{3} r^2) \quad  &(r < R_c)
\end{cases}
\]
\[ V_{m} = \begin{cases}
\frac{4}{3} \pi G \rho (R_m^3 - R_c^3) \frac{1}{r} \quad &(r > R_m) \\
2 \pi G \rho (R_m^2 - \frac{1}{3} r^2) - \frac{4}{3} \pi G \rho R_c^3 \frac{1}{r} \quad &(R_c < r < R_m) \\
2 \pi G \rho (R_m^2 - R_c^2) \quad &(r < R_c)
\end{cases}
\]
\begin{equation*}
V = V_c + V_m
\end{equation*}
\[ a_{c} = \begin{cases}
\frac{4}{3} \pi G \rho R_c^3 \frac{1}{r} \quad &(r > R_c)\\
2 \pi G \rho (R_c^2 - \frac{1}{3} r^2) \quad  &(r < R_c)
\end{cases}
\]
\[ a_{m} = \begin{cases}
- \frac{4}{3} \pi G \rho (R_m^3 - R_c^3) \frac{1}{r^2} \quad &(r > R_m) \\
- \frac{4}{3} \pi G \rho (r^3 - R_c^3) \frac{1}{r^2} \quad &(R_c < r < R_m) \\
0 \quad \quad \quad \quad \quad \quad \quad \quad \quad &(r < R_c)
\end{cases}
\]
\newline
\begin{equation*}
V = V_c + V_m \quad a = a_c + a_m
\end{equation*}
Die Darstellung von Potential und Anziehung unter dieser vereinfachten Modell. 

\newpage
\section{PREM density model of the Earth}

\end{document}